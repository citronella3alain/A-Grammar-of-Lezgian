As was mentioned in section \ref{sec:intro}, Lezgian has eighteen cases in total, four grammatical and fourteen local (74). The four grammatical cases are the absolutive, ergative, genitive, and the dative (74). The local cases are divided into five localizations: Ad, Post, Sub, Super, In, each of which has three locatives: Essive, Elative, and Directive (74). As Lezgian is missing the indirective case, there are only 14 combinations(74). The tables below depict the singular inflections of \textit{sew} (``bear''), first with the grammatical cases, then with the local cases:
\begin{center}
    \begin{tabular}{|l|l|l|}
        \hline
        Case Name & Lezgian Noun Form & English Translation \\ \hhline{|=|=|=|}
        Absolutive & sew &the bear \\
        Ergative & sew-re &the bear \\
        Genitive & sew-re-n &of the bear \\
        Dative & sew-re-z &to the bear \\ \hline
        Adessive & sew-re-w &at the bear \\
        Adelative & sew-re-w-aj &from the bear \\
        Addirective & sew-re-w-di & toward the bear \\ \hline
        Postessive & sew-re-$q^h$ & behind the bear \\
        Postelative & sew-re-$q^h$-aj & from behind the bear \\
        Postdirective & sew-re-$q^h$-di & to behind the bear \\ \hline
        Subessive & sew-re-k & under the bear \\
        Subelative & sew-re-k-aj & from under the bear \\
        Subdirective & sew-re-k-adi & to under the bear \\ \hline
        Superessive & sew-re-l & on the bear \\
        Superelative & sew-re-l-aj & off the bear \\
        Superdirective & sew-re-l-di & onto the bear \\ \hline
        Inessive & sew-re & in the bear \\
        Inelative & sew-r\"{a}j & out of the bear \\ \hline
    \end{tabular}
\end{center}
It is also worth noting the existence of the oblique stem on which all cases but the absolutive are based on (4). For example, in the table of examples above, the oblique stem appears identical to the ergative case, i.e. as \textit{sew-re}; the ergative case is a null suffix attached to the oblique stem (14).

As mentioned in section \ref{sec:intro} and seen in the table above, Lezgian is an ergative-absolutive language. However, the flexible word order and lack of agreement in Lezgian makes it difficult to assign morphological rules that determine grammatical case, i.e. Absolutive, Ergative, or Dative, to any particular structural position. For example, as alluded in \ref{subsec:subject}, both ergative agents and dative experiencers tend to precede absolutive arguments, thus implying that one of them is likely to jump to the TP specifier position, the flexibility of Lezgian word order also produces examples where the absolutive argument has to jump to the TP specifier position in order to satisfy the X' rules, c.f. Figure \ref{fig:sent4a}.

Lezgian's agglutinative morphology, however, simplifies the process of isolating morphemes and, in the context of the locative cases, the analysis of the locative cases as prepositional phrases: except for the inessive and inelative cases, each locative case is marked by a suffix that is added to the oblique form(meaningless form identical to ergative from which all non-absolutie forms are derived). This locative suffix serves as the P head for a prepositional phrase while the oblique suffix acts as a D head in a determiner phrase. This is illustrated in example \ref{sent:ex2a} where \textit{-kaj}, the subelative case marker, serves as the P head for the Prepositional Phrase and \textit{di}, the oblique stem, as the D head. Similarly, in example \ref{sent:ex2b}, \textit{laj}, the superelative case marker, serves as the P head for the prepositional phrase and \textit{di} serves as the D head again. Both of these examples are illustrated in Figure \ref{fig:ex2}.
