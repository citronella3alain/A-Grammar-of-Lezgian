\documentclass[12pt, letterpaper]{article}
\usepackage[utf8]{inputenc}
\usepackage[margin=1in]{geometry}
\usepackage{float}
\usepackage{forest, leipzig, linguex, flexisym}
\usepackage[normalem]{ulem}
\usepackage{amssymb, textcomp}
\forestset{qtree/.style={for tree={parent anchor=south, 
           child anchor=north,align=center,inner sep=0pt}}}
\setlength{\marginparwidth}{2cm}
\usepackage{todonotes}
\usepackage{hyperref}


\title{A Grammar of Lezgian}
\author{Allen Mao}

\makeglossaries
\newleipzig{aop}{aop}{aorist participle}
\newleipzig{aori}{aor}{aorist}
\newleipzig{hort}{hort}{hortative}
\newleipzig{imc}{imc}{imperfective converb}
\newleipzig{impf}{impf}{imperfective}
\newleipzig{inel}{inel}{inelative case}
\newleipzig{inessc}{iness}{inessive case}
\newleipzig{masd}{msd}{masdar}
\newleipzig{pt}{pt}{particle}
\newleipzig{sbelc}{sbel}{subelative case}
\newleipzig{sbstz}{sbst}{substantivizer}
\newleipzig{srelc}{srel}{superelative case}

\begin{document}

\maketitle
%\tableofcontents
\section{Introduction}
This paper presents an X' analysis of the Lezgian language, first with X' phrase structure rules that describe a generative grammar for the major lexical categories, e.g. VP (verb phrase), DP (determiner phrase), NP (noun phrase), etc. These rules then serve the foundation for a further discussion on more interesting topics in Section \ref{sec:spec-topics}, mainly X, Y, and Z\todo[]{List here}.

The Lezgian language is spoken in a region of about 5000 $km^2$ between the Eastern Caucasus mountains and the Caspian Sea, present-day southern Daghestan, Russia and northern Azerbaijan (16). Haspelmath calculates that with a 90\% retention rate and over 466 000 Lezgians in the 1989 census of the Soviet Union, there are well over 400 000 speakers of Lezgian (16). Lezgian is part of the Lezgic branch of the Nakho-Daghestanian language family, commonly known as the ``North-East Caucasion" or ``East Caucasian" language family (1). Other languages from the Nakho-Daghestanian language family include the Chechen and Ingush from the Nakh branch and Avar, an Avaric language (1). Although the economic and political dominance of the Russian language has clearly contributed to the decline of the Lezgian language, Lezgian remains taught at several levels of education in the Republic of Daghestan and Lezgian-language publications, radio broadcasting, and theater productions remain available (24). In addition, Lezgian remains vibrant, if not dominant, in rural areas of the Republic of Daghestan (24). As such, Haspelmath concludes that ``as long as the Lezgians remain in their traditional settlement areas, Lezgian is not an endangered language" (24).

As a head-final language, i.e. heads mostly follow complements, the dominant word order of Lezgian is SOV (subject-object-verb), although other word orders do surface especially when spoken (5). Head-finality is compulsory, however, in noun phrases (NP), adjective phrases (AdjP), and postpositional phrases (PP) (5).

Lezgian clauses are uniformly ergative and Lezgian morphology is agglutinative (4-5). There exist 36 cases depending on number (Singular, Plural), case (Absolutive, Ergative, Genetive, Dative, Essive, Elative, Directive), and localization (Ad, Sub, Post, Super, In) (4). Case and localizations occur together, e.g. ``Ad" and ``Essive" together form the "Adessive" case. Lezgian also exhibits no agreement in noun phrases or finite verbs (6).
\section{Phrase Structure}
\subsection{Verb Phrase}
All Lezgian verbs take between one and three arguments, inclusive (268-269). As per Lezgian's Ergative-Absolutive paradigm, most Lezgian verb valence patterns contain an absolutive argument (269). This absolutive argument always acts as a theme (268, 270). Each of such intransitive valence patterns also has a corresponding transitive valence pattern of the same arguments plus an Ergative argument (269). This Ergative argument always functions as the agent (270). Lezgian verbs can also accept Locative or Dative arguments, the former serving as recipients and experiencers and the latter as a local case. While there is weak evidence that both Ergative agents and Dative experiencers precede the Absolutive argument, word order is flexible in Lezgian (294-295). Lezgian verbs also lack subject-verb agreement (294), but the free word order suggests head movement.

\begin{center}
    \begin{tabular}{r@{\hskip3pt}lr}
        VP &\textrightarrow (DP+)(PP+)(AdvP+)V' &(adjuncts) \\
        V' &\textrightarrow (DP+)(PP+)V &(complements)
    \end{tabular}
\end{center}

\ex. First Person Absolutive Pronoun, \textit{Zun}: c.f. Figure \ref{fig:ex5} \\ 
    \gll\label{sent:ex5}Zun ata-na. \\
        I:\Abs[] come-\Aori[] \\
        ``I came. (251)
    \z.


\subsection{Auxiliary Verb Phrase and Tense Phrase}
Lezgian's agglutinative nature makes possible to isolate the affixes that indicate tense and aspect. Refer to the examples in glosses 

The phrase structure rules are written as follows:
\begin{center}
    \begin{tabular}{r@{\hskip3pt}l}
        TP &\textrightarrow DP T'  \\
        T' &\textrightarrow $V_{AUX}P$ T \\
        $V_{AUX}P$ &\textrightarrow $V_{AUX}'$ \\
        $V_{AUX}'$ &\textrightarrow VP $V_{AUX}$ \\
    \end{tabular}
\end{center}

As such, it is possible to distinguish $V_{\text{AUX}}$ heads from T heads.
\subsection{Subjects}
\label{subsec:subject}
As a role-dominated ergative-absolutive language, the concept of the subject in Lezgian is less straight-forward than it is in a reference-dominated nominative-accusative language like English (294). While ergative agents and dative experiencers may both be subjects as they precede absolutive arguments in unmarked order, the flexible word order of Lezgian makes this evidence very weak (294-295). Instead, the evidence for subjects in Lezgian comes from omissions in coreferential constructions just as they are in English (295). For example, in the English sentence \textit{Maria promised Kim to meet Hans}, the subject of \textit{to meet} and \textit{to promise} corefer to \textit{Maria} (295). In this case, the former serves as the target and the latter as the controller of the omission (295). Just like in English, Lezgian subjects often serve as controllers (295). Example sentence \ref{sent:ex8a} shows how the ergative argument behaves as the subject as its omitted in the dependent clause just as example sentences \ref{sent:ex8b} and \ref{sent:ex8c} do for absolutive and dative arguments.
\subsection{Complementizer Phrase}
\label{subsec:cp}
Whereas Lezgian declarative sentences do not exhibit an overt complementizer, polar, i.e. yes-or-no, questions are expressed in Lezgian with the interrogative mood verb affix ``-ni" (417). This suffix can be classified as the complementizer in question formation (c.f. gloss \ref{sent:ex4a}). Similarly, when a question asks for a selection among a number of options, e.g. \textit{Professor.di ktab k'el-zawa-j-di j-ani ja ta\^{x}ajt'a k\^{x}i-zwa-j-di ja-ni?} (``Is the professor reading or writing a book?"), the suffix ``-ni" follows both verbs (418); this situation can be modelled as a conjunction phrase joining the two complementizer phrases where each conjunction phrase is headed by the ``-ni" suffix (c.f. gloss \ref{sent:ex4b}). 

Unlike questions in English, Lezgian parametric questions do not move an interrogative phrase to a clause-initial position (421). Instead, the questioned constituent is subsituted with an interrogative pronoun (421). These parametric questions bear the conditional verb suffix \textit{-t'a}. This suffix serves as the complementizer for parametric questions.

\ex. \label{sent:ex4}Question Formation, c.f. Figure \ref{fig:sent4a}, \ref{fig:sent4b} \\
    \ag.\label{sent:ex4a}Betxoven.a-n muzyka wa-z k'an-da-ni? \\
    Beethoven-\Gen[] music you-\Dat[] like-\Fut[]-\Q[] \\
    ``Do you like Beethoven's music?" (417)
    \bg.\label{sent:ex4b}Professor.di ktab k'el-zawa-j-di j-ani ja ta\^{x}ajt'a k\^{x}i-zwa-j-di ja-ni? \\
    professor(\Erg[]) book read-\Impf[]-\Ptcp[]-\Sbstz[] \Cop[]-\Q[] or or write-\Impf[]-\Ptcp[]-\Sbstz[] \Cop[]-\Q[] \\
    ``Is the professor reading or writing a book?" (418)

Given this information and Lezgian's right-headedness, we write the CP rules as follows:
\begin{center}
    \begin{tabular}{r@{\hskip3pt}l}
        CP &\textrightarrow C'  \\
        C' &\textrightarrow TP C
    \end{tabular}
\end{center}
\subsection{Noun and Determiner Phrases}
\label{sec:np}
Haspelmath describes a noun phrase as one of the following (252):
\begin{enumerate}
    \item a pronoun
    \item a noun head with optional preceding modifiers, i.e. quantifiers, demonstratives, adjective phrases, Genitive NPs, relative clauses
    \item a nominalized clause
\end{enumerate}
This paper modifies Haspelmath's categorization to fit the X' schema with the notion of the Determiner Phrase (DP). Pronouns are recategorized as Determiner heads, but unlike the X'-schema for English, quantifiers and demonstratives are categorized as N's.

As modifiers precede the head noun, the X' rules for the NP are as follows:
\begin{center}
    \begin{tabular}{r@{\hskip3pt}l}
        NP &\textrightarrow N'  \\
        N' &\textrightarrow (AdjP+) N' \\
        N' &\textrightarrow N
    \end{tabular}
\end{center}

Noun phrases in Lezgian do not allow postpositional phrases, $NP$s in oblique cases, or adverbs as adjunct modifiers (252). As such, English NPs that would normally require a PP adjunct such as ``stories about collective farm life" (c.f. gloss \ref{sent:ex2a} and \ref{sent:ex2b}) would employ a relative clause for the same meaning (252). 

Just as in English syntax, pronouns are determiners (c.f. gloss \ref{sent:ex1}. 
%However, Lezgian determiners are otherwise different from English determiners. Lezgian lacks a definite article, i.e. has a null definite article, and although the numeral \textit{sa}, ``one" is used an indefinite article, it is optional. As seen by the demonstrative combination \textit{ha i-} (that this) in gloss \ref{sent:ex6a} and the co-occurence between the genitive pronoun and the demonstrative, demonstratives are not determiners. Instead, they are adjectives in Lezgian.
However, as we see in gloss \ref{sent:ex6a}, since the demonstratives \textit{ha} (``that") and \textit{i} (``this") can form the combination ``ha i" (this) (190-191), demonstratives can co-occur and therefore cannot be determiners. This analysis is further supported by gloss \ref{sent:ex6b}, a construction that would be illegal under English syntax. This gloss reveals that demonstratives are actually of category N', similar to the English ``one". Refer to \todo{do trees} for the tree representations of these glosses. 

Lezgian marks indefiniteness but not definiteness, as seen in gloss \ref{sent:ex6a}. However, what Haspelmath identifies as an optional indefinite article (230), \textit{sa} (``one"), is instead classified as an adjective in this paper. In its place, the oblique case endings fill the role as a determiner head when present. This classification allows for an analysis that preserves Lezgian's right-headedness while assigning a syntactic role for the oblique case markings. 

The X' rules for the DP are thus summarized as follows:
\begin{center}
    \begin{tabular}{r@{\hskip3pt}l}
        DP &\textrightarrow (DP) D'  \\
        D' &\textrightarrow (NP) D
    \end{tabular}
\end{center}

As shown in gloss \ref{sent:ex1}, genitive DPs serve as specifiers to the DP they modify.



%Quantifiers such as \textit{(sa) t'imil} ([a] little, [a] few) \textit{(sa) b\"{a}zi} (some, several), \textit{sa \v{s}umud} (several), \textit{sa q'adar} (a certain amount), \textit{gzaf} (a lot), \textit{xejlin} (a lot), \textit{iq'wan} (so much, so many) (253) are classified here as adjectives rather than determiners. Refer to the next section for a discussion on this choice.

\subsection{Postpositional Phrase}
Postpositions are the only type of adposition in Lezgian and are often derived from spatial adverbs, spatial nouns, or converbial verb forms (213). In addition to these postpositions that Haspelmath identifies, this paper will also consider the endings of the locative cases, i.e. essive, elative, and directive with ad, sub, post, super, and in localizations, as prepositions. An example of the former is shown in the glosses of \ref{sent:ex3}. Glosses \ref{sent:ex2a} and \ref{sent:ex2b} demonstrate the latter. 

\ex. \label{sent:ex3}Postpositions\\
    \ag.\label{sent:ex3a}Kac stol.di-n k'anik aka\^{x}-na. \\
        cat table-\Gen[] under enter-\Aori[] \\
        ``The cat went under the table." (170) 
    \bg.\label{sent:ex3b}Kac stol.di-n k'anikaj xkec'-na. \\
        cat table-\Gen[] from.under go.out-\Aori[] \\
        ``The cat came out from under the table." (170)

The X' rules are summarized as follows:
\begin{center}
    \begin{tabular}{r@{\hskip3pt}l}
        PP &\textrightarrow P'  \\
        P' &\textrightarrow DP P 
    \end{tabular}
\end{center}


\subsection{Trees}
\label{subsec:trees}
\begin{figure}[H]
    \centering
\begin{forest}, baseline, qtree
    [T'
        [$V_{\text{AUX}}P$
            [$V_{\text{AUX}}'$
                [VP [V' [V \\ fiz \\ go]]
                ]
                [$V_{\text{AUX}}$,name=src]
            ]
        ]
        [$V_{\text{AUX}}T$\\ zwa-$\varnothing$ \\ \Impf,name=dest]
    ]
\draw[->] (src) to[out=north east, in=south west] (dest);
\end{forest}
    \caption{Imperfective affirmative}
    \label{fig:sent5a}
\end{figure}
\begin{figure}[H]
    \centering
\begin{forest}, baseline, qtree
    [T'
        [$V_{\text{AUX}}P$
            [$V_{\text{AUX}}'$
                [VP [V' [V \\ fiz \\ go]]
                ]
                [$V_{\text{AUX}}$,name=src]
            ]
        ]
        [$V_{\text{AUX}}T$\\ zwa-\v{c} \\ \Impf-\Neg,name=dest]
    ]
\draw[->] (src) to[out=north east, in=south west] (dest);
\end{forest}
    \caption{Imperfective Negative}
    \label{fig:sent5b}
\end{figure}
\begin{figure}[H]
    \centering
\begin{forest}, baseline, qtree
    [T'
        [$V_{\text{AUX}}P$
            [$V_{\text{AUX}}'$
                [VP [V' [V \\ fiz \\ go]]
                ]
                [$V_{\text{AUX}}$,name=src]
            ]
        ]
        [$V_{\text{AUX}}T$\\ zwa-j \\ \Impf-\Pst,name=dest]
    ]
\draw[->] (src) to[out=north east, in=south west] (dest);
\end{forest}
    \caption{Past Imperfective Affirmative}
    \label{fig:sent5c}
\end{figure}
\begin{figure}[H]
    \centering
\begin{forest}, baseline, qtree
[CP[C'
    [TP
        [DP,name=tgt
        [DP [D'
                [NP [N' [N \\ Betxoven \\ Beethoven]
                                ]]
            [D \\ .a-n \\ \Gen]
                ]]
                    [D'
                        [NP [N'[N \\ muzyka \\ music]]]
                        [D \\ $\varnothing_D$]
                    ]
        ]
        [T'
            [VP
                [DP
                        [D'[D \\ wa-z \\ you-\Dat]]
                ]
                [V'
                    [\sout{DP},name=src]
                    [V \\ k'an \\ like]
                ]
            ]
            [T \\ -da- \\ \Fut]
        ]
    ]
    [C \\ -ni \\ \Q]
]]
\draw[->] (src) to[out=north, in=east] (tgt);
\end{forest}
    \caption{Complete tree for gloss \ref{sent:ex4a}}
    \label{fig:sent4a}
\end{figure}
\begin{figure}[H]
    \centering
\begin{forest}, baseline, qtree
[CP
    [CP
        [C'
            [TP
                [DP,name=newprofdp
                    [D'
                        [NP
                            [N'
                                [N \\ Professor \\ professor]
                            ]
                        ]
                        [D \\ di-$\varnothing$ \\ \Obl \Erg]
                    ]
                ]
                [T'
                    [VP
                        [V'
                            [DP
                                [D'
                                    [NP
                                        [TP
                                            [T'
                                                [$V_{AUX}P$
                                                    [$V_{AUX}'$
                                                        [VP
                                                            [\sout{DP}]
                                                            [V'
                                                                [DP [D' [NP[N'[N \\ ktab]]][D]]]
                                                                [V \\ k'el \\ read]
                                                            ]
                                                        ]
                                                        [$V_{AUX}$ \\ zawa \\ \Impf]
                                                    ]
                                                ]
                                                [T \\ j \\ \Ptcp]
                                            ]
                                        ]
                                        [N'
                                            [N \\ di \\ \Sbstz]
                                        ]
                                    ]
                                    [D]
                                ]
                            ]
                            [\sout{V}]
                        ]
                    ]
                    [\sout{V-T}]
                ]
            ]
            [V-T-C \\ ja-$\varnothing$-ni \\ \Cop-$\varnothing$-\Q]
        ]
    ]
    [Conj \\ ja ta\^{x}ajt'a \\ or]
    [CP [C'
        [TP
            [T'
                [VP
                    [V'
                        [DP
                            [D'
                                [NP
                                    [TP
                                        [T'
                                            [$V_{AUX}P$
                                                [$V_{AUX}'$
                                                    [VP
                                                        [V'
                                                            [V  \\ k\^{x}i \\ write]
                                                        ]
                                                    ]
                                                    [$V_{AUX}$ \\ zawa \\ \Impf]
                                                ]
                                            ]
                                            [T \\ j \\ \Ptcp]
                                        ]
                                    ]
                                    [N'
                                        [N \\ di \\ \Sbstz]
                                    ]
                                ]
                                [D \\ $\varnothing_D$]
                            ]
                        ]
                        [V]
                    ]
                ]
                [V-T]
            ]
        ]
        [V-T-C \\ \Cop-$\varnothing$-\Q]
    ]]
]
\end{forest}
    \caption{Caption}
    \label{fig:my_label}
\end{figure}
\begin{figure}[H]
    \centering
\begin{forest}, baseline, qtree
[TP
    [DP 
        [DP [D' [D \\ Zi \\ I:\Gen]] ] 
        [D' 
            [NP 
                [N' 
                    [AdjP [Adj' [Adj \\ \v{g}we\v{c}'i \\ little]]]
                    [N' [N \\ wax \\ sister]]
                ]
            ] 
            [D \\ $\varnothing_D$] 
        ]
    ]
    [T'
        [VP 
            [V' [V \\ ata \\ come]]
        ]
        [T \\ na \\ \Aori]
    ]
]
\end{forest}
    \caption{Complete tree for gloss \ref{sent:ex1}}
    \label{fig:ex1}
\end{figure}
\begin{figure}[H]
    \centering
\begin{forest}, baseline, qtree
[DP [D' [NP [N'
    [CP [C'
        [TP [T'
            [VP
                [V'
                    [PP [P'
                        [DP
                            [DP
                                [D' [NP [N' [N \\ kolkhoz \\ collective farm]]]
                                    [D \\ di-n \\ \Gen]
                                ]
                            ]
                            [D'
                                [NP [N' [N \\ ja\v{s}aji\v{s} \\ life]]]
                                [D \\ di \\ \Obl]
                            ]
                        ]
                    [P \\ -kaj \\ \Sbelc]]]
                    [V' [V \\ k\^{x}e \\ write]]
                ]
            ]
            [T \\ -nwa- \\ \Prf]
        ]]
        [C \\ -j \\ \Ptcp]
    ]] 
    [N'[N\\ rasskaz-ar \\ story-\Pl]]]][D \\ $\varnothing_D$]]]
\end{forest}
\caption{Complete tree for gloss \ref{sent:ex2a}}
\label{fig:ex2a}
\end{figure}
\begin{figure}[H]
    \centering
\begin{forest}, baseline, qtree
[DP [D'
    [NP [N'
        [AdjP [Adj'
            [PP [P'
                [DP [D'
                        [NP
                            [N' [N \\ revolucija \\ revolution]]
                        ]
                        [D \\ di \\ \Obl]
                ]]
                [P \\ laj \\ \Srelc]
            ]]
            [Adj \\ wikikan \\ previous]
        ]]
        [N'[N \\ \"{u}m\"{u}r \\ life]]
    ]]
    [D  \\ $\varnothing_D$]
]]
\end{forest}
    \caption{Complete Tree for gloss \ref{sent:ex2b}}
    \label{fig:ex2b}
\end{figure}
\begin{figure}[H]
    \centering
    \begin{forest}, baseline, qtree
[TP
    [DP
        [D'
            [NP
                [N'
                    [N \\ kac \\ cat]
                ]
            ]
            [D \\ $\varnothing_D$]
        ]
    ]
    [T'
        [VP
            [V'
                [PP
                    [P'
                        [DP 
                            [D'
                                [NP
                                    [N'
                                        [N \\ stol \\ table]
                                    ]
                                ]
                                [D'
                                    [D \\ di-n \\ \Gen]
                                ]
                            ]
                        ]
                        [P \\ k'anik \\ under]
                    ]
                ]
                [V'
                    [V \\ aka\^{x} \\ enter]
                ]
            ]
        ]
        [T \\ -na \\ \Aori]
    ]
]
    \end{forest}
    \caption{Complete Tree for gloss \ref{sent:ex3a}}
    \label{fig:ex3a}
\end{figure}
\begin{figure}[H]
    \centering
    \begin{forest}, baseline, qtree
[TP
    [DP
        [D'
            [NP
                [N'
                    [N \\ kac \\ cat]
                ]
            ]
            [D \\ $\varnothing_D$]
        ]
    ]
    [T'
        [VP
            [V'
                [PP
                    [P'
                        [DP 
                            [D'
                                [NP
                                    [N'
                                        [N \\ stol \\ table]
                                    ]
                                ]
                                [D'
                                    [D \\ di-n \\ \Gen]
                                ]
                            ]
                        ]
                        [P \\ k'anikaj \\ from.under]
                    ]
                ]
                [V'
                    [V \\ xkec' \\ go.out]
                ]
            ]
        ]
        [T \\ -na \\ \Aori]
    ]
]
    \end{forest}
    \caption{Complete tree for gloss \ref{sent:ex3b}}
    \label{fig:sent3b}
\end{figure}


\section{Special Topics}
\label{sec:spec-topics}
\subsection{Questions}
As mentioned in section \ref{subsec:cp}, polar questions in Lezgian are formed with the interrogative verb suffix ``-ni'', c.f. example \ref{sent:ex4a} and \ref{sent:ex4c} (417). A similar structure is used for questions that ask for a selection among a number of the choices, e.g. \textit{Professordi ktab k'elzawajdi ja\textbf{ni} ja ta\^{x}ajt'a k\^{x}izwajdi ja\textbf{ni}?} (``Is the professor reading or writing a book?'') (c.f. example \ref{sent:ex4b} for glossing) (418).

\ex. Question Types: Polar, Binary-Choice, Constituent, Embdedded Polar, Embedded Constituent, Embedded Alternative:  \\
    \ag.\label{sent:ex4c}Farid ata-na-ni? \\ 
    Farid come-\Aori-\Q \\
    ``Has Farid come?" (7)
    \bg.\label{sent:ex4d}Farid mus ata-na \\
    Farid when come-\Aori \\
    ``When did Farid come?'' (8)
    \bg.\label{sent:ex4e}Wun hi\^{x}tin dust.uni-n muq'uw mus fe-na? \\
    you:\Abs[] which friend-\Gen[] to when go-\Aori[] \\
    ``When did you go to which friend?'' (425)
    \bg.\label{sent:ex4f}Za sadra, kkal.i xa-nwa-t'a, akwa-n. \\
    I:\Erg[] \Pt[] [cow(\Erg[]) bear-\Prf[]-\Cond[]] see-\Hort[] \\
    ``Let me see whether the cow has calved.'' (425)
    \bg.\label{sent:ex4g}Jarab abur.u wu\v{c} luhu-zwa-t'a? \\
    \Pt[] they(\Erg[]) what:\Abs[] say-\Impf[]-\Cond[] \\
    ``I wonder what they are saying.'' (427)
    \bg.\label{sent:ex4h}Ada-z im axwar ja-ni, xabar ja-ni \v{c}i-zwa-\v{c}-ir. \\
    he-\Dat[] [this:\Abs[] dream \Cop[]-\Q[] news \Cop[]-\Q[]] know-\Impf[]-\Neg[]-\Pst[] \\
    ``He did not know whether this was dream or reality.'' (426)
    \z.

Lezgian content (Wh-) questions, on the other hand, do not exhibit any verb affix and instead substitute the constituent in question with the associated interrogative pronoun, e.g. \textit{mus}(``when'') in \textit{Farid mus atana?} (``When did Farid come?'') (c.f. example \ref{sent:ex4d} for gloss) (8). One single content question may also contain multiple interrogative pronouns, c.f. example \ref{sent:ex4e}. A comparison between the tree structures of polar question \textit{Farid atana\textbf{ni}?} (``Has Farid come?'') in Figure \ref{fig:sent4b} and the content question \textit{Farid mus atana?} (``When did Farid come?'') in Figure \ref{fig:sent4d} shows how verb hopping may produce both structures and how a C-head integrates in both polar and content questions. 

Embedded questions in Lezgian, both polar, e.g. \textit{Za sadra, kkali xanwa\textbf{t'a} akwan}(``Let me see whether the cow has calved.''), and parametric, e.g. \textit{Jarab aburu wu\v{c} luhuzwa\textbf{t'a}?}(``I wonder what they are saying.''), are marked by the conditional suffix \textit{-t'a} on the verb (426). This serves as the C-head for polar and embedded questions as seen in glossed examples \ref{sent:ex4f} and \ref{sent:ex4g}. Embedded choice questions, e.g. \textit{Adaz im axwar ja\textbf{ni}, xabar ja\textbf{ni} \v{c}izwa\v{c}ir}(``He did not know whether this was dream or reality'') exhibit a similar structure where the C-head ``-ni'' affixes to both verbs, as glossed in example \ref{sent:ex4h} (426).

As seen in these examples, word order in Lezgian is very flexible save its verb finality. While the absolutive argument precedes the dative argument in example \ref{sent:ex4a} (tree \todo{reference tree}), in example \ref{sent:ex4b} (tree XX), the ergative TP specifier precedes the absolutive argument; few word order restrictions are enforced in Lezgian and the order of arguments and adjuncts in Lezgian clauses is very free (298). This paper accounts for this flexibility with movement, but this very flexibility suggests that this movement is not systematic. 
\subsection{Case/Agreement}
As was mentioned in section \ref{sec:intro}, Lezgian has eighteen cases in total, four grammatical and fourteen local (74). The four grammatical cases are the absolutive, ergative, genitive, and the dative (74). The local cases are divided into five localizations: Ad, Post, Sub, Super, In, each of which has three locatives: Essive, Elative, and Directive (74). As Lezgian is missing the indirective case, there are only 14 combinations(74). The tables below depict the singular inflections of \textit{sew} (``bear''), first with the grammatical cases, then with the local cases:
\begin{center}
    \begin{tabular}{|l|l|l|}
        \hline
        Case Name & Lezgian Noun Form & English Translation \\ \hhline{|=|=|=|}
        Absolutive & sew &the bear \\
        Ergative & sew-re &the bear \\
        Genitive & sew-re-n &of the bear \\
        Dative & sew-re-z &to the bear \\ \hline
        Adessive & sew-re-w &at the bear \\
        Adelative & sew-re-w-aj &from the bear \\
        Addirective & sew-re-w-di & toward the bear \\ \hline
        Postessive & sew-re-$q^h$ & behind the bear \\
        Postelative & sew-re-$q^h$-aj & from behind the bear \\
        Postdirective & sew-re-$q^h$-di & to behind the bear \\ \hline
        Subessive & sew-re-k & under the bear \\
        Subelative & sew-re-k-aj & from under the bear \\
        Subdirective & sew-re-k-adi & to under the bear \\ \hline
        Superessive & sew-re-l & on the bear \\
        Superelative & sew-re-l-aj & off the bear \\
        Superdirective & sew-re-l-di & onto the bear \\ \hline
        Inessive & sew-re & in the bear \\
        Inelative & sew-r\"{a}j & out of the bear \\ \hline
    \end{tabular}
\end{center}
It is also worth noting the existence of the oblique stem on which all cases but the absolutive are based on (4). For example, in the table of examples above, the oblique stem appears identical to the ergative case, i.e. as \textit{sew-re}; the ergative case is a null suffix attached to the oblique stem (14).

As mentioned in section \ref{sec:intro} and seen in the table above, Lezgian is an ergative-absolutive language. However, the flexible word order and lack of agreement in Lezgian makes it difficult to assign morphological rules that determine grammatical case, i.e. Absolutive, Ergative, or Dative, to any particular structural position. For example, as alluded in \ref{subsec:subject}, both ergative agents and dative experiencers tend to precede absolutive arguments, thus implying that one of them is likely to jump to the TP specifier position, the flexibility of Lezgian word order also produces examples where the absolutive argument has to jump to the TP specifier position in order to satisfy the X' rules, c.f. Figure \ref{fig:sent4a}.

Lezgian's agglutinative morphology, however, simplifies the process of isolating morphemes and, in the context of the locative cases, the analysis of the locative cases as prepositional phrases: except for the inessive and inelative cases, each locative case is marked by a suffix that is added to the oblique form(meaningless form identical to ergative from which all non-absolutie forms are derived). This locative suffix serves as the P head for a prepositional phrase while the oblique suffix acts as a D head in a determiner phrase. This is illustrated in example \ref{sent:ex2a} where \textit{-kaj}, the subelative case marker, serves as the P head for the Prepositional Phrase and \textit{di}, the oblique stem, as the D head. Similarly, in example \ref{sent:ex2b}, \textit{laj}, the superelative case marker, serves as the P head for the prepositional phrase and \textit{di} serves as the D head again. Both of these examples are illustrated in Figure \ref{fig:ex2}.

\section{Abbreviations}
\printglossaries
\section{References}
\begin{thebibliography}{}
\bibitem{hasp}Haspelmath, M. (2011). \textit{A grammar of Lezgian (Vol. 9).} Walter de Gruyter.

\end{thebibliography}
\end{document}
