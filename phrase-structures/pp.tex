Postpositions are the only type of adposition in Lezgian and are often derived from spatial adverbs, spatial nouns, or converbial verb forms (213). In addition to these postpositions that Haspelmath identifies, this paper will also consider the endings of the locative cases, i.e. essive, elative, and directive with ad, sub, post, super, and in localizations, as prepositions. An example of the former is shown in the glosses of \ref{sent:ex3}. Glosses \ref{sent:ex2a} and \ref{sent:ex2b} demonstrate the latter. 



The X' rules are summarized as follows:
\begin{center}
    \begin{tabular}{r@{\hskip3pt}l}
        PP &\textrightarrow P'  \\
        P' &\textrightarrow DP P 
    \end{tabular}
\end{center}