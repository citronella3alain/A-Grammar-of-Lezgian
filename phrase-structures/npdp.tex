Haspelmath describes a noun phrase as one of the following (252):
\begin{enumerate}
    \item a pronoun
    \item a noun head with optional preceding modifiers, i.e. quantifiers, demonstratives, adjective phrases, Genitive NPs, relative clauses
    \item a nominalized clause
\end{enumerate}
This paper modifies Haspelmath's categorization to fit the X' schema with the notion of the Determiner Phrase (DP). Pronouns are recategorized as Determiner heads, but unlike the X'-schema for English, quantifiers and demonstratives are categorized as N's.

As modifiers precede the head noun, the X' rules for the NP are as follows:
\begin{center}
    \begin{tabular}{r@{\hskip3pt}l}
        NP &\textrightarrow N'  \\
        N' &\textrightarrow (AdjP+) N' \\
        N' &\textrightarrow N
    \end{tabular}
\end{center}

Noun phrases in Lezgian do not allow postpositional phrases, $NP$s in oblique cases, or adverbs as adjunct modifiers (252). As such, English NPs that would normally require a PP adjunct such as ``stories about collective farm life" (c.f. gloss \ref{sent:ex2a} and \ref{sent:ex2b}) would employ a relative clause for the same meaning (252). 

\ex. Relative clause, Adjective based on preposition: c.f. Figure \ref{fig:ex2} \\
    \ag.\label{sent:ex2a}[Kolxoz.di-n ja\v{s}aji\v{s}.di-kaj k\^{x}e-nwa-j] rasskaz-ar \\
        [kolkhoz-\Gen[] life-\Sbelc[] write-\Prf[]-\Ptcp[]] story-\Pl[] \\
        ``stories about collective farm life" (252)
    \bg.\label{sent:ex2b}revoljucija.di-laj wilikan \"{u}m\"{u}r \\
        revolution-\Obl[]-\Srelc[] previous life \\
        ``life before the revolution" (252)
    \z.

The tree representations of examples \ref{sent:ex2a} and \ref{sent:ex2b} are shown below:
\begin{figure}[H]
\centering
\begin{minipage}{.4\textwidth}
    \scalebox{.8}{
        \begin{forest}, baseline, qtree
            [DP [D' [NP [N'
                [CP [C'
                    [TP [T'
                        [VP
                            [V'
                                [PP [P'
                                    [DP
                                        [DP
                                            [D' [NP [N' [N \\ kolkhoz \\ collective farm]]]
                                                [D \\ di-n \\ \Gen]
                                            ]
                                        ]
                                        [D'
                                            [NP [N' [N \\ ja\v{s}aji\v{s} \\ life]]]
                                            [D \\ di \\ \Obl]
                                        ]
                                    ]
                                [P \\ -kaj \\ \Sbelc]]]
                                [V' [V \\ k\^{x}e \\ write]]
                            ]
                        ]
                        [T \\ -nwa- \\ \Prf]
                    ]]
                    [C \\ -j \\ \Ptcp]
                ]] 
                [N'[N\\ rasskaz-ar \\ story-\Pl]]]][D \\ $\varnothing_D$]]]
        \end{forest}   
    }
\end{minipage}
\hfill\begin{minipage}{.4\textwidth}
    \begin{forest}, baseline, qtree
        [DP [D'
            [NP [N'
                [AdjP [Adj'
                    [PP [P'
                        [DP [D'
                                [NP
                                    [N' [N \\ revolucija \\ revolution]]
                                ]
                                [D \\ di \\ \Obl]
                        ]]
                        [P \\ laj \\ \Srelc]
                    ]]
                    [Adj \\ wikikan \\ previous]
                ]]
                [N'[N \\ \"{u}m\"{u}r \\ life]]
            ]]
            [D  \\ $\varnothing_D$]
        ]]
        \end{forest}
\end{minipage}
\caption{Complete tree for gloss \ref{sent:ex2a} and \ref{sent:ex2b}}
\label{fig:ex2}
\end{figure}
Just as in English syntax, pronouns are determiners (c.f. gloss \ref{sent:ex1}. 

\ex.\label{sent:ex1}Absolutive and Gentitive Forms of 1\Sg[] pronoun: c.f. Figure \ref{fig:ex1} for tree \\ 
    \ag.Zun ata-na. \\
        I:\Abs[] come-\Aori[] \\
        ``I came. (251)
    \bg. [Zi [\v{g}we\v{c}'i wax]] ata-na. \\
        [I:\Gen[] [little sister]] came-\Aori[] \\
        ``My little sister came. (251)
    \z.

However, as we see in gloss \ref{sent:ex6a}, since the demonstratives \textit{ha} (``that") and \textit{i} (``this") can form the combination ``ha i" (this) (190-191), demonstratives can co-occur and therefore cannot be determiners. This analysis is further supported by gloss \ref{sent:ex6b}, a construction that would be illegal under English syntax. This gloss reveals that demonstratives are actually of category N', similar to the English ``one". 

Lezgian marks indefiniteness but not definiteness, as seen in gloss \ref{sent:ex6a}. However, what Haspelmath identifies as an optional indefinite article (230), \textit{sa} (``one"), is instead classified as an adjective in this paper. In its place, the oblique case endings fill the role as a determiner head when present. This classification allows for an analysis that preserves Lezgian's right-headedness while assigning a syntactic role for the oblique case markings. 

\ex. Demonstratives co-occuring with each other or determiners \\
    \ag.\label{sent:ex6a}Ha i klass.d-a sa mus jat'ani za-ni k'el-na-j. \\
    that this class-\Inessc[] one when \Indf[] I:\Erg[]-also study-\Aori[]-\Pst[] \\
    ``At one time I, too, was a student in this classroom." (191)
    \bg.\label{sent:ex6b}i zi kic' \\
    this I:\Gen[] dog \\
    ``this my dog" (not: \textit{*dog of this I}) (261)
    \z.

The X' rules for the DP are thus summarized as follows:
\begin{center}
    \begin{tabular}{r@{\hskip3pt}l}
        DP &\textrightarrow (DP) D'  \\
        D' &\textrightarrow (NP) D
    \end{tabular}
\end{center}

As shown in gloss \ref{sent:ex1}, genitive DPs serve as specifiers to the DP they modify. 

\begin{figure}[H]
    \centering
    \begin{minipage}{.4\textwidth}
        \scalebox{.75}{
            \begin{forest}, baseline, qtree
                [TP
                    [T'
                        [$V_{AUX}P$
                            [$V_{AUX}'$
                                [VP
                                    [V'
                                        [DP
                                            [D' [D \\ Zun \\ 1\Sg.\Abs]]
                                        ]
                                        [\sout{V} \\ \sout{ata},name=v]
                                    ]
                                ]
                                [\sout{$V_{AUX}$} \\ \sout{na},name=vaux]
                            ]
                    ]
                    [V-$V_{AUX}$-T \\ ata-na \\ come-\Aori-$\varnothing$,name=t]
                ]]
                \draw[->](v)to[out=north east, in=south](vaux);
                \draw[->](vaux)to[out=south east, in=south](t);
            \end{forest}
        }
    \end{minipage}
    \begin{minipage}{.4\textwidth}
        \scalebox{.6}{
            \begin{forest}, baseline, qtree
                [TP
                    [T'
                        [$V_{AUX}P$ [$V_{AUX}'$
                            [VP
                                [DP 
                                    [DP [DP [D' [D \\ ]]]
                                        [D' [D \\ Zi \\ I:\Gen]] 
                                        ] 
                                        [D' 
                                            [NP 
                                                [N' 
                                                    [AdjP [Adj' [Adj \\ \v{g}we\v{c}'i \\ little]]]
                                                    [N' [N \\ wax \\ sister]]
                                                ]
                                            ] 
                                    [D \\ $\varnothing_D$] 
                                ]
                            ]
                                [V' [\sout{V},name=v]]
                            ]
                            [\sout{V-$V_{AUX}$},name=vaux]
                        ]]
                        [V-$V_{AUX}$-T \\ ata-na \\ come-\Aori-$\varnothing$,name=t]
                    ]
                ]
                \draw[->] (v) to[out=north east, in=south] (vaux);
                \draw[->] (vaux) to[out=north east, in=south] (t);
            \end{forest}    
        }
    \end{minipage}
    \caption{Tree for gloss \ref{sent:ex1}}
    \label{fig:ex1}
\end{figure}

