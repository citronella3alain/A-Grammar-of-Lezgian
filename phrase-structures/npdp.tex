Haspelmath describes a noun phrase as one of the following (252):
\begin{enumerate}
    \item a pronoun
    \item a noun head with optional preceding modifiers, i.e. quantifiers, demonstratives, adjective phrases, Genitive NPs, relative clauses
    \item a nominalized clause
\end{enumerate}
This paper modifies Haspelmath's categorization to fit the X' schema with the notion of the Determiner Phrase (DP). Pronouns are recategorized as Determiner heads, but unlike the X'-schema for English, quantifiers and demonstratives are categorized as N's.

As modifiers precede the head noun, the X' rules for the NP are as follows:
\begin{center}
    \begin{tabular}{r@{\hskip3pt}l}
        NP &\textrightarrow N'  \\
        N' &\textrightarrow (AdjP+) N' \\
        N' &\textrightarrow N
    \end{tabular}
\end{center}

Noun phrases in Lezgian do not allow postpositional phrases, $NP$s in oblique cases, or adverbs as adjunct modifiers (252). As such, English NPs that would normally require a PP adjunct such as ``stories about collective farm life" (c.f. gloss \ref{sent:ex2a} and \ref{sent:ex2b}) would employ a relative clause for the same meaning (252). 

Just as in English syntax, pronouns are determiners (c.f. gloss \ref{sent:ex1}. 
%However, Lezgian determiners are otherwise different from English determiners. Lezgian lacks a definite article, i.e. has a null definite article, and although the numeral \textit{sa}, ``one" is used an indefinite article, it is optional. As seen by the demonstrative combination \textit{ha i-} (that this) in gloss \ref{sent:ex6a} and the co-occurence between the genitive pronoun and the demonstrative, demonstratives are not determiners. Instead, they are adjectives in Lezgian.
However, as we see in gloss \ref{sent:ex6a}, since the demonstratives \textit{ha} (``that") and \textit{i} (``this") can form the combination ``ha i" (this) (190-191), demonstratives can co-occur and therefore cannot be determiners. This analysis is further supported by gloss \ref{sent:ex6b}, a construction that would be illegal under English syntax. This gloss reveals that demonstratives are actually of category N', similar to the English ``one". Refer to \todo{do trees} for the tree representations of these glosses. 

Lezgian marks indefiniteness but not definiteness, as seen in gloss \ref{sent:ex6a}. However, what Haspelmath identifies as an optional indefinite article (230), \textit{sa} (``one"), is instead classified as an adjective in this paper. In its place, the oblique case endings fill the role as a determiner head when present. This classification allows for an analysis that preserves Lezgian's right-headedness while assigning a syntactic role for the oblique case markings. 

The X' rules for the DP are thus summarized as follows:
\begin{center}
    \begin{tabular}{r@{\hskip3pt}l}
        DP &\textrightarrow (DP) D'  \\
        D' &\textrightarrow (NP) D
    \end{tabular}
\end{center}

As shown in gloss \ref{sent:ex1}, genitive DPs serve as specifiers to the DP they modify.



%Quantifiers such as \textit{(sa) t'imil} ([a] little, [a] few) \textit{(sa) b\"{a}zi} (some, several), \textit{sa \v{s}umud} (several), \textit{sa q'adar} (a certain amount), \textit{gzaf} (a lot), \textit{xejlin} (a lot), \textit{iq'wan} (so much, so many) (253) are classified here as adjectives rather than determiners. Refer to the next section for a discussion on this choice.
