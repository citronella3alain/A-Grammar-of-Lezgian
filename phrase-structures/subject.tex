As a role-dominated ergative-absolutive language, the concept of the subject in Lezgian is less straight-forward than it is in a reference-dominated nominative-accusative language like English (294). While ergative agents and dative experiencers may both be subjects as they precede absolutive arguments in unmarked order, the flexible word order of Lezgian makes this evidence very weak (294-295). Instead, the evidence for subjects in Lezgian comes from omissions in coreferential constructions just as they are in English (295). For example, in the English sentence \textit{Maria promised Kim to meet Hans}, the subject of \textit{to meet} and \textit{to promise} corefer to \textit{Maria} (295). In this case, the former serves as the target and the latter as the controller of the omission (295). Just like in English, Lezgian subjects often serve as controllers (295). Example sentence \ref{sent:ex8a} shows how the ergative argument behaves as the subject as its omitted in the dependent clause just as example sentences \ref{sent:ex8b} and \ref{sent:ex8c} do for absolutive and dative arguments.

\ex.\label{sent:ex8}Coreferential Omission \\
    \ag.\label{sent:ex8a}Ada jar\v{g}-ar.i-z kilig-un dawamar-na. \\
    he(\Erg[]) [distance-\Pl[]-\Dat[] look-\Masd[]] continue-\Aori[] \\
    ``He kept looking into the distance.'' (295)
    \bg.\label{sent:ex8b}G\"{u}ldeste wiri \v{z}\"{u}re.di-n k'walax-ar awu-n.i-z ma\v{z}bur \^{x}a-na. \\
    G\"{u}ldeste [all kind-\Gen[] work-\Pl[] do-\Masd[]-\Dat[]] forced become-\Aori[] \\
    ``G\"{u}ldeste was forced to do work of all kinds.'' (295)
    \cg.\label{sent:ex8c}Wa-z k\"{u}\v{c}e.di-z fi-n.i-kaj ki\v{c}e-zwa-ni? \\
    you-\Dat[] [street-\Dat[] go-\Masd[]-\Sbelc[]] afraid-\Impf[]-\Q[] \\
    ``Are you afraid to go on the street?''
    \z.

