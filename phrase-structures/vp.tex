All Lezgian verbs take between one and three arguments, inclusive (268-269). As per Lezgian's Ergative-Absolutive paradigm, most Lezgian verb valence patterns contain an absolutive argument (269). This absolutive argument always acts as a theme (268, 270). Each of such intransitive valence patterns also has a corresponding transitive valence pattern of the same arguments plus an Ergative argument (269). This Ergative argument always functions as the agent (270). Lezgian verbs can also accept Locative or Dative arguments, the former serving as recipients and experiencers and the latter as a local case. While there is weak evidence that both Ergative agents and Dative experiencers precede the Absolutive argument, word order is flexible in Lezgian (294-295). Lezgian verbs also lack subject-verb agreement (294), but the free word order suggests head movement.
\begin{center}
    \begin{tabular}{r@{\hskip3pt}lr}
        VP &\textrightarrow (DP+)(PP+)(AdvP+)V' &(adjuncts) \\
        V' &\textrightarrow (DP+)(PP+)V &(complements)
    \end{tabular}
\end{center}