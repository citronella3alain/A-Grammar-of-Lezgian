Whereas Lezgian declarative sentences do not exhibit an overt complementizer, polar, i.e. yes-or-no, questions are expressed in Lezgian with the interrogative mood verb affix ``-ni" (417). This suffix can be classified as the complementizer in question formation (c.f. gloss \ref{sent:ex4a}). Similarly, when a question asks for a selection among a number of options, e.g. \textit{Professor.di ktab k'el-zawa-j-di j-ani ja ta\^{x}ajt'a k\^{x}i-zwa-j-di ja-ni?} (``Is the professor reading or writing a book?"), the suffix ``-ni" follows both verbs (418); this situation can be modelled as a conjunction phrase joining the two complementizer phrases where each conjunction phrase is headed by the ``-ni" suffix (c.f. gloss \ref{sent:ex4b}). 

Unlike questions in English, Lezgian wh-questions do not move an interrogative phrase to a clause-initial position (421). Instead, the questioned constituent is subsituted with an interrogative pronoun (421). These wh-questions bear the conditional verb suffix \textit{-t'a}. This suffix serves as the complementizer for wh-questions.


Given this information and Lezgian's right-headedness, we write the CP rules as follows:
\begin{center}
    \begin{tabular}{r@{\hskip3pt}l}
        CP &\textrightarrow C'  \\
        C' &\textrightarrow TP C
    \end{tabular}
\end{center}