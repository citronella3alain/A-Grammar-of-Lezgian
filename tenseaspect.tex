Lezgian verbs inflect according to six tense-aspect categories: imperfective future, aorist, perfect, contiuative, and past. This section focuses on the first four, i.e. the simple tenses, as the latter two combine with the simple tenses (140). As was mentioned in Subsection \ref{subsec:auxt}, it is possible to isolate the aspect and tense into auxiliary verbs ($V_{AUX}$) and tense heads (T), respectively. Whereas the former describe the relationship between the event time (ET), when the event took place, and the reference time (RT), the time from which the event is described or referenced, the latter describes the relationship between the reference time and the utterance time (UT), the time at which the statement is made.
\subsubsection{Imperfective}
In Lezgian, the imperfective(\Impf[]) refers to processes happening at the time of reference (140); the event time is the same as the reference time, i.e. ET=RT. \todo[inline]{Include reference to example \ref{sent:ex9}}. As seen in examples \ref{sent:ex7} and \ref{sent:ex4b}, the imperfective marker can co-occur with the past tense marker or the participle affix, thus indicating that the imperfective is a $V_{AUX}$ head.
\subsubsection{Future}
In Lezgian, the colloquial use of the future is to refer to future situations, although in formal styles, it also refers to habitual situations. The former situation describes the relation UT > RT, the same future tense as in English. This is illustrated in example \ref{sent:ex10a}. The latter, however, seems to suggest that when used alone, the future embeds ET=RT, as seen in example \ref{sent:ex10b}. \todo[inline]{How about the Beethoven example? c.f. \ref{sent:ex4a}}
\subsubsection{Aorist}
Although Haspelmath describes the aorist as a reference to perfective events in the past, this paper analyzes the aorist more as an aspect than a tense since the aorist can co-occur with the the past tense, c.f. example \ref{sent:ex6a}. The perfective nature therefore implies that the event time precedes the reference time, i.e. ET > RT. To distinguish the aorist from the past aorist and the perfective (see next subsubsection), it appears that RT = UT serves as a default tense, rather than present tense, for the aorist.

As was just alluded to, the past aorist follows ET > RT > UT for situations in the remote past and situations that no longer have an effect (143).
\subsubsection{Perfective}
\label{subsec:perf}
Like the aorist, the perfect refers to events in the past, but the perfect generally refers to past events with current relevance (143)