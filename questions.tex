As mentioned in section \ref{subsec:cp}, polar questions in Lezgian are formed with the interrogative verb suffix ``-ni'', c.f. example \ref{sent:ex4a} and \ref{sent:ex4c} (417). A similar structure is used for questions that ask for a selection among a number of the choices, e.g. \textit{Professordi ktab k'elzawajdi ja\textbf{ni} ja ta\^{x}ajt'a k\^{x}izwajdi ja\textbf{ni}?} (``Is the professor reading or writing a book?'') (c.f. example \ref{sent:ex4b} for glossing) (418).

\ex. Question Types: Polar, Binary-Choice, Constituent, Embdedded Polar, Embedded Constituent, Embedded Alternative:  \\
    \ag.\label{sent:ex4c}Farid ata-na-ni? \\ 
    Farid come-\Aori-\Q \\
    ``Has Farid come?" (7)
    \bg.\label{sent:ex4d}Farid mus ata-na \\
    Farid when come-\Aori \\
    ``When did Farid come?'' (8)
    \bg.\label{sent:ex4e}Wun hi\^{x}tin dust.uni-n muq'uw mus fe-na? \\
    you:\Abs[] which friend-\Gen[] to when go-\Aori[] \\
    ``When did you go to which friend?'' (425)
    \bg.\label{sent:ex4f}Za sadra, kkal.i xa-nwa-t'a, akwa-n. \\
    I:\Erg[] \Pt[] [cow(\Erg[]) bear-\Prf[]-\Cond[]] see-\Hort[] \\
    ``Let me see whether the cow has calved.'' (425)
    \bg.\label{sent:ex4g}Jarab abur.u wu\v{c} luhu-zwa-t'a? \\
    \Pt[] they(\Erg[]) what:\Abs[] say-\Impf[]-\Cond[] \\
    ``I wonder what they are saying.'' (427)
    \bg.\label{sent:ex4h}Ada-z im axwar ja-ni, xabar ja-ni \v{c}i-zwa-\v{c}-ir. \\
    he-\Dat[] [this:\Abs[] dream \Cop[]-\Q[] news \Cop[]-\Q[]] know-\Impf[]-\Neg[]-\Pst[] \\
    ``He did not know whether this was dream or reality.'' (426)
    \z.

Lezgian content (Wh-) questions, on the other hand, do not exhibit any verb affix and instead substitute the constituent in question with the associated interrogative pronoun, e.g. \textit{mus}(``when'') in \textit{Farid mus atana?} (``When did Farid come?'') (c.f. example \ref{sent:ex4d} for gloss) (8). One single content question may also contain multiple interrogative pronouns, c.f. example \ref{sent:ex4e}. A comparison between the tree structures of polar question \textit{Farid atana\textbf{ni}?} (``Has Farid come?'') in Figure \ref{fig:sent4b} and the content question \textit{Farid mus atana?} (``When did Farid come?'') in Figure \ref{fig:sent4d} shows how verb hopping may produce both structures and how a C-head integrates in both polar and content questions. 

Embedded questions in Lezgian, both polar, e.g. \textit{Za sadra, kkali xanwa\textbf{t'a} akwan}(``Let me see whether the cow has calved.''), and parametric, e.g. \textit{Jarab aburu wu\v{c} luhuzwa\textbf{t'a}?}(``I wonder what they are saying.''), are marked by the conditional suffix \textit{-t'a} on the verb (426). This serves as the C-head for polar and embedded questions as seen in glossed examples \ref{sent:ex4f} and \ref{sent:ex4g}. Embedded choice questions, e.g. \textit{Adaz im axwar ja\textbf{ni}, xabar ja\textbf{ni} \v{c}izwa\v{c}ir}(``He did not know whether this was dream or reality'') exhibit a similar structure where the C-head ``-ni'' affixes to both verbs, as glossed in example \ref{sent:ex4h} (426).

As seen in these examples, word order in Lezgian is very flexible save its verb finality. While the absolutive argument precedes the dative argument in example \ref{sent:ex4a} (tree \todo{reference tree}), in example \ref{sent:ex4b} (tree XX), the ergative TP specifier precedes the absolutive argument; few word order restrictions are enforced in Lezgian and the order of arguments and adjuncts in Lezgian clauses is very free (298). This paper accounts for this flexibility with movement, but this very flexibility suggests that this movement is not systematic. 