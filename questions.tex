As mentioned in section \ref{sec:cp}, polar questions in Lezgian are formed with the interrogative verb suffix ``-ni'', e.g.\textit{Betxovenan muzyka waz k'anda\textbf{ni}?} (``Do you like Beethoven's music?") or \textit{Farid atana\textbf{ni}?} (``Has Farid come?'') (c.f. example \ref{sent:ex4a} and \ref{sent:ex4c}, respectively, for examples glossed) (417). A similar structure is used for questions that ask for a selection among a number of the choices, e.g. \textit{Professordi ktab k'elzawajdi ja\textbf{ni} ja ta\^{x}ajt'a k\^{x}izwajdi ja\textbf{ni}?} (``Is the professor reading or writing a book?'') (c.f. example \ref{sent:ex4b} for glossing) (418).

As seen in these examples, word order in Lezgian is very flexible save its verb finality. While the absolutive argument precedes the dative argument in example \ref{sent:ex4a} (tree \todo{reference tree}), in example \ref{sent:ex4b} (tree XX), the ergative TP specifier precedes the absolutive argument; few word order restrictions are enforced in Lezgian and the order of arguments and adjuncts in Lezgian clauses is very free (298). This paper accounts for this flexibility with verb movement, but this very flexibility suggests that this movement is not systematic. 

Lezgian content (Wh-) questions, on the other hand, do not exhibit any verb affix and instead substitute the constituent in question with the associated interrogative pronoun, e.g. \textit{mus}(``when'') in \textit{Farid mus atana?} (``When did Farid come?'') (c.f. example \ref{sent:ex4d} for gloss) (8). One single content question may also contain multiple interrogative pronouns, c.f. example \ref{sent:ex4e}. A comparison between the tree structures of polar question \textit{Farid atana\textbf{ni}?} (``Has Farid come?'') in Figure \ref{fig:sent4b} and the content question \textit{Farid mus atana?} (``When did Farid come?'') in Figure \ref{fig:sent4d} show how verb hopping may produce both structures and how a C-head integrates in both polar and content questions. 
