This paper presents an X' analysis of the Lezgian language, first with X' phrase structure rules that describe a generative grammar for the major lexical categories, e.g. VP (verb phrase), DP (determiner phrase), NP (noun phrase), etc. These rules then serve the foundation for a further discussion on more interesting topics in Section \ref{sec:spec-topics}, mainly X, Y, and Z\todo[]{List here}.

The Lezgian language is spoken in a region of about 5000 $km^2$ between the Eastern Caucasus mountains and the Caspian Sea, present-day southern Daghestan, Russia and northern Azerbaijan (16). Haspelmath calculates that with a 90\% retention rate and over 466 000 Lezgians in the 1989 census of the Soviet Union, there are well over 400 000 speakers of Lezgian (16). Lezgian is part of the Lezgic branch of the Nakho-Daghestanian language family, commonly known as the ``North-East Caucasion" or ``East Caucasian" language family (1). Other languages from the Nakho-Daghestanian language family include the Chechen and Ingush from the Nakh branch and Avar, an Avaric language (1). Although the economic and political dominance of the Russian language has clearly contributed to the decline of the Lezgian language, Lezgian remains taught at several levels of education in the Republic of Daghestan and Lezgian-language publications, radio broadcasting, and theater productions remain available (24). In addition, Lezgian remains vibrant, if not dominant, in rural areas of the Republic of Daghestan (24). As such, Haspelmath concludes that ``as long as the Lezgians remain in their traditional settlement areas, Lezgian is not an endangered language" (24).

As a head-final language, i.e. heads mostly follow complements, the dominant word order of Lezgian is SOV (subject-object-verb), although other word orders do surface especially when spoken (5). Head-finality is compulsory, however, in noun phrases (NP), adjective phrases (AdjP), and postpositional phrases (PP) (5).

Lezgian clauses are uniformly ergative and Lezgian morphology is agglutinative (4-5). There exist 36 cases depending on number (Singular, Plural), case (Absolutive, Ergative, Genetive, Dative, Essive, Elative, Directive), and localization (Ad, Sub, Post, Super, In) (4). Case and localizations occur together, e.g. ``Ad" and ``Essive" together form the "Adessive" case. Lezgian also exhibits no agreement in noun phrases or finite verbs (6).