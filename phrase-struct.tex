\subsection{Verb Phrase}
As per Lezgian's Ergative-Absolutive paradigm, most Lezgian verb valence patterns contain an absolutive argument and each intransitive valence pattern with an absolutive argument has a corresponding transitive valence pattern of the same arguments plus an Ergative argument (269). Additionally, all Lezgian verbs take no fewer than one argument and no more than three arguments (268-269). While there is weak evidence that both Ergative agents and Dative experiencers precede the Absolutive argument, word order is flexible in Lezgian (294-295). Lezgian verbs also lack subject-verb agreement (294), but the free word order suggests head movement. Verbs in Lezgian also have a strong correspondance between semantic roles and case marking (270). 
\begin{center}
    \begin{tabular}{r@{\hskip3pt}l}
        VP &\textrightarrow (DP+)(PP+)(AdvP+)V' (adjuncts) \\
        V' &\textrightarrow (DP+)(PP+)V (complements)
    \end{tabular}
\end{center}
\subsection{Auxiliary Verb Phrase and Tense Phrase}
Lezgian's agglutinative nature means that it is possible to isolate the affixes that indicate tense and aspect.

The phrase structure rules are written as follows:
\begin{center}
    \begin{tabular}{r@{\hskip3pt}l}
        TP &\textrightarrow DP T'  \\
        T' &\textrightarrow $V_{AUX}P$ T \\
        $V_{AUX}P$ &\textrightarrow $V_{AUX}'$ \\
        $V_{AUX}'$ &\textrightarrow VP $V_{AUX}$ \\
    \end{tabular}
\end{center}
\ex. Selected Verb Conjugations of \textit{fin} "go" (127). Trees for a, b, c represented in Figures \ref{fig:sent5a}, \ref{fig:sent5b}, and \ref{fig:sent5c} \\
    \ag.\label{sent:5a}fi-zwa \\
    go-\Impf[] \\
    \bg.\label{sent:5b}fi-zwa-\v{c} \\
    go-\Impf[]-\Neg[] \\
    \cg.\label{sent:5c}fi-zwa-j \\
    go-\Impf[]-\Pst[] \\
    \dg. fe-nwa \\
    go-\Prf[] \\

As such, it is possible to distinguish $V_{\text{AUX}}$ heads from T heads.
\subsection{Complementizer Phrase}
\label{sec:cp}
Whereas declarative sentences do not exhibit an overt complementizer, polar questions are expressed in Lezgian with the interrogative mood verb affix ``-ni" (417). In other words, Lezgian questions are CPs When the question offers a choice, e.g. ``Is the professor reading or writing a book?", the suffix ``-ni" follows both verbs (418); the two complementizer phrases are joined together in a conjunction phrase. Lezgian parametric questions do not move an interrogative phrase to a clause-initial position; any constituent can be questioned.

\ex. \label{sent:ex4}Question Formation, c.f. Figure \ref{fig:sent4a} \\
    \ag.\label{sent:ex4a}Betxoven.a-n muzyka wa-z k'an-da-ni? \\
    Beethoven-\Gen[] music you-\Dat[] like-\Fut[]-\Q[] \\
    ``Do you like Beethoven's music?" (417)
    \bg.\label{sent:ex4b}Professor.di ktab k'el-zawa-j-di j-ani ja ta\^{x}ajt'a k\^{x}i-zwa-j-di ja-ni? \\
    professor(\Erg[]) book read-\Impf[]-\Ptcp[]-\Sbstz[] \Cop[]-\Q[] or or write-\Impf[]-\Ptcp[]-\Sbstz[] \Cop[]-\Q[] \\
    ``Is the professor reading or writing a book?" (418)

Given this information and Lezgian's right-headedness, we write the CP rules as follows:
\begin{center}
    \begin{tabular}{r@{\hskip3pt}l}
        CP &\textrightarrow C'  \\
        C' &\textrightarrow TP C
    \end{tabular}
\end{center}
\subsection{Noun and Determiner Phrases}
\label{sec:np}
Haspelmath describes a noun phrase as one of the following (252):
\begin{enumerate}
    \item a pronoun
    \item a noun head with optional preceding modifiers, i.e. quantifiers, demonstratives, adjective phrases, Genitive NPs, relative clauses
    \item a nominalized clause
\end{enumerate}
This paper will modify Haspelmath's categorization to fit the X' schema with the notion of the Determiner Phrase ($DP$). Pronouns are recategorized as Determiner heads, but unlike the X'-schema for English, quantifiers and demonstratives are categorized as Noun heads.

As modifiers precede the head noun, the X' rules for the NP are as follows:
\begin{center}
    \begin{tabular}{r@{\hskip3pt}l}
        NP &\textrightarrow N'  \\
        N' &\textrightarrow (AdjP+) N' \\
        N' &\textrightarrow N
    \end{tabular}
\end{center}
Noun phrases in Lezgian do not allow postpositional phrases, $NP$s in oblique cases, or adverbs as adjunct modifiers (252). As such, English NPs that would normally require a PP adjunct such as ``stories about collective farm life" (c.f. gloss \ref{sent:ex2a} and \ref{sent:ex2b}) would employ a relative clause for the same meaning (252). 

Just as in English syntax, pronouns are determiners (c.f. gloss \ref{sent:ex1} and \ref{sent:ex5}). 
%However, Lezgian determiners are otherwise different from English determiners. Lezgian lacks a definite article, i.e. has a null definite article, and although the numeral \textit{sa}, ``one" is used an indefinite article, it is optional. As seen by the demonstrative combination \textit{ha i-} (that this) in gloss \ref{sent:ex6a} and the co-occurence between the genitive pronoun and the demonstrative, demonstratives are not determiners. Instead, they are adjectives in Lezgian.
However, as we see in gloss \ref{sent:ex6a}, since the demonstratives \textit{ha} (``that") and \textit{i} (``this") can form the combination ``ha i" (this) (190-191), demonstratives can co-occur and therefore cannot be determiners. This analysis is further supported by gloss \ref{sent:ex6b}, a construction that would be illegal under English syntax. This gloss reveals that demonstratives are actually of category N', similar to the English ``one". Refer to \todo{do trees} for the tree representations of these glosses. 

Lezgian marks indefiniteness but not definiteness, as seen in gloss \ref{ex6a}. However, what Haspelmath identifies as an optional indefinite article (230), \textit{sa} (``one"), is instead classified as an adjective in this paper. In its place, the oblique case endings fill the role as a determiner head when present. This classification allows for an analysis that preserves Lezgian's right-headedness while assigning a syntactic role for the oblique case markings. 

The X' rules for the DP are thus summarized as follows:
\begin{center}
    \begin{tabular}{r@{\hskip3pt}l}
        DP &\textrightarrow (DP) D'  \\
        D' &\textrightarrow (NP) D
    \end{tabular}
\end{center}

As shown in gloss \ref{sent:ex1}, genitive DPs serve as specifiers to the DP they modify.

\ex. DP with DP specifier: c.f. Figure \ref{fig:ex1} \\ 
    \gll\label{sent:ex1}[Zi [\v{g}we\v{c}'i wax]] ata-na. \\
        [I:\Gen[] [little sister]] came-\Aori[] \\
        ``My little sister came. (251)
    %\z.

\ex. Relative clause, Adjective based on preposition: c.f. Figures \ref{fig:ex2a} and \ref{sent:ex2b} \\
    \ag.\label{sent:ex2a}[Kolxoz.di-n ja\v{s}aji\v{s}.di-kaj k\^{x}e-nwa-j] rasskaz-ar \\
        [kolkhoz-\Gen[] life-\Sbelc[] write-\Prf[]-\Ptcp[]] story-\Pl[] \\
        ``stories about collective farm life" (252)
    \bg.\label{sent:ex2b}revoljucija.di-laj wilikan \"{u}m\"{u}r \\
        revolution-\Srelc[] previous life \\
        ``life before the revolution" (252)
    \z.
    
\ex. First Person Absolutive Pronoun, \textit{Zun} \\ 
    \gll\label{sent:ex5}Zun ata-na. \\
        I:\Abs[] come-\Aori[] \\
        ``I came. (251)
    \z.
    
\ex. Demonstratives co-occuring with each other or determiners \\
    \ag.\label{sent:ex6a}Ha i klass.d-a sa mus jat'ani za-ni k'el-na-j. \\
    that this class-\Inessc[] one when \Indf[] I:\Erg[]-also study-\Aori[]-\Pst[] \\
    ``At one time I, too, was a student in this classroom." (191)
    \bg.\label{sent:ex6b}i zi kic' \\
    this I:\Gen[] dog \\
    ``this my dog" (not: \textit{*dog of this I}) (261)

%Quantifiers such as \textit{(sa) t'imil} ([a] little, [a] few) \textit{(sa) b\"{a}zi} (some, several), \textit{sa \v{s}umud} (several), \textit{sa q'adar} (a certain amount), \textit{gzaf} (a lot), \textit{xejlin} (a lot), \textit{iq'wan} (so much, so many) (253) are classified here as adjectives rather than determiners. Refer to the next section for a discussion on this choice.



\subsection{Postpositional Phrase}
Postpositions are the only type of adposition in Lezgian and are often derived from spatial adverbs, spatial nouns, or converbial verb forms (213). In addition to these postpositions that Haspelmath identifies, this paper will also consider the endings of the locative cases, i.e. essive, elative, and directive with ad, sub, post, super, and in localizations, as prepositions. An example of the former is shown in the glosses of \ref{sent:ex3}. Glosses \ref{sent:ex2a} and \ref{sent:ex2b} demonstrate the latter. 

\ex. \label{sent:ex3}Postpositions\\
    \ag.\label{sent:ex3a}Kac stol.di-n k'anik aka\^{x}-na. \\
        cat table-\Gen[] under enter-\Aori[] \\
        ``The cat went under the table." (170) 
    \bg.\label{sent:ex3b}Kac stol.di-n k'anikaj xkec'-na. \\
        cat table-\Gen[] from.under go.out-\Aori[] \\
        ``The cat came out from under the table." (170)

We summarize the X' rules as follows:
\begin{center}
    \begin{tabular}{r@{\hskip3pt}l}
        PP &\textrightarrow P'  \\
        P' &\textrightarrow DP P 
    \end{tabular}
\end{center}

